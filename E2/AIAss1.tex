\documentclass[11pt,a4paper]{article}
\usepackage{amsmath}

\begin{document}
\section*{2.1}
\subsection*{a}
A nonlinear transfer function gives the neural network a universal property: Given enough layers and neurons, the network can model any function within a certain accuracy. In a network with a linear transfer function we can only compute a linear function. A network with a linear transfer function of $n$ layers will always be equivalent to a network with only one layer: linear functions can always be concatinated.\\
Whenever the function that we are trying to model isn't a linear function, it is useful to use a nonlinear transfer function.
Examples include image classification or speech recognition.
\subsection*{b}
Consider a simple neural network with two input neurons that can both either be 0 or 1 and one output layer. We want to construct an AND gate with our network, so without bias our quest would be to find $w = (w_1, w_2)$ such that:
\begin{align*}
  0 \leq 0\\
  w_1 \leq 0\\
  w_2 \leq 0\\
  w_1 + w_2 > 0\\
\end{align*}
which is impossible. We can easily however create the network with a bias, if we have the weights $w = (1, 1)$ and the bias
$\theta = \frac{3}{2}$. Then $sgn(w^Tx - \theta)$ would give us AND.
\subsection*{c}
\subsection*{d}
\section*{2.2}
\section*{2.3}
\end{document}
